\documentclass[11pt]{article}
\usepackage[portrait,margin=0.8in,tmargin=1.0in]{geometry}
\usepackage{longtable}

\begin{document}
%\begin{tiny}

\section{AFLOW$\pi$ Configuration file}
\begin{center}
\begin{tabular}{|p{3cm}|l|p{9cm}|c|}
\hline
\textbf{Variable Name} & \textbf{Type} & \textbf{Description} & \textbf{Req?}\\
\hline


%\Xhline{2\arrayrulewidth}
%\textbf{Engine Runtime} & \multicolumn{4}{l}{}\\
%\hline

work\_dir   & Text & Path of the directory where AFLOW$\pi$ will generate the directory tree. Typically this path will be in a scratch partition.& Y\\
\hline
pseudo\_dir & Text & Path of the directory containing the pseudopotential files you want to use in your calculations& Y \\
\hline
engine\_dir & Text & Path of the directory containing the executables of your calculation engine& Y \\
\hline
gipaw\_dir  & Text & Path of the directory containing the executables. Required if workflow contains steps that use GIPAW. & D \\
\hline
pao\_dir    & Text & Path of the directory containing the gaussian basis files used in ACBN0. Required if workflow contains ACBN0.& D\\
\hline
fig\_dir    & Text & Path of directory where figures will be copied to. If omiitted, figures will only be generated in the directory tree.& N\\
\hline
engine                & Text & Type of calculation engine. Only engine currently suppported is Quantum Espresso& N \\
\hline
exec\_prefix          & Text & Runtime execution prefix. ex. mpirun -n 64& N\\
\hline
exec\_postfix         & Text & Runtime execution postfix. ex. -npool 4& N  \\
\hline
exec\_prefix\_serial  & Text & Overrides exec\_postfix for serial executables. On some computing clusters certain executables do not run properly with parallel execution prefixes& N\\
\hline
exec\_postfix\_serial & Text & Overrides exec\_postfix for serial executables. When using exec\_prefix\_serial it is sometimes necessary to also override the exec\_postfix& N \\
\hline
exec\_postfix\_nscf   & Text & Overrides exec\_postfix for the nscf step of ACBN0 or the generation of PAO-TB hamiltonians& N \\
\hline

%\end{tabular}


%\begin{tabular}{|p{3cm}|l|p{5.5cm}|c|p{5.5cm}|}
%\hline
%\textbf{Variable Name} & \textbf{Type} & \textbf{Description} & \textbf{Req?} &  \textbf{Notes}\\
%\hline
%\hline
%\textbf{Data Directories} & \multicolumn{4}{l}{} \\
%\hlyine


%\end{tabular}
%\begin{tabular}{|p{3cm}|l|p{5.5cm}|c|p{5.5cm}|}
%\hline
%\textbf{Variable Name} & \textbf{Type} & \textbf{Description} & \textbf{Req?} &  \textbf{Notes}\\
%\hline
%\hline
%\textbf{Cluster and Queue} & \multicolumn{4}{l}{} \\
%\hline
queue             & Text & Name of queue to send PBS jobs to. On some cluster computers this is needed. -q value is appended to the qsub command.& D \\
\hline
job\_template     & Text &Path of the cluster job submission template. Required if submitting calculations to a queue.& D\\
\hline
steps\_as\_jobs   & Bool & If True: Each step in the calculation workflow will be submitted as a new job. Can be useful in certain situations like a very long workflow.& N \\
\hline
restart\_buffer   & Float & Amount of buffer time to include in diring the runtime of the workflow. The buffer allows the calculation engine to gracefully exit before hitting the walltime requested for the cluster job. The job will automatically resubmit itself and restart where it left off. If the value is less than 1.0 then it the buffer will be a (1-value)*walltime requested for the job. If the value is greater than 1.0 then the buffer will be that number of seconds. Default is 0.9& N\\
\hline

\end{tabular}
\newpage
\begin{tabular}{|p{3cm}|l|p{9cm}|c|}
\hline
\textbf{Variable Name} & \textbf{Type} & \textbf{Description} & \textbf{Req?}\\
\hline
%\hline
%\textbf{Data Povenance} & \multicolumn{4}{l}{}\\
%\hline
cluster\_type     & Text& Type of cluster submission/queue system. Required if submitting jobs to a queue. if not included the calculations will run locally.& D\\
\hline
author         & Text & Name of user & N \\
\hline
affiliation    & Text & Affiliation of user& \\
\hline
title          & Text & Title of project& N\\
\hline

%\end{tabular}
%\begin{tabular}{|p{3cm}|l|p{5.5cm}|c|p{5.5cm}|}
%\hline
%\textbf{Variable Name} & \textbf{Type} & \textbf{Description} & \textbf{Req?} &  \textbf{Notes}\\
%\hline
%\hline
%\textbf{Miscellaneous}  & \multicolumn{4}{l}{}\\
%\hline
log\_level       & Text & Logging level of AFLOW$\pi$. Acceptable values: CRITICAL, ERROR, WARNING, INFO, DEBUG. Default is INFO & N\\
\hline
plot\_file\_type & Text & File extension of generated figures. Acceptable values: .png and .pdf. default is .png.& N \\
\hline
pseudo\_regex    & Text & Custom regular expression used to choose a pseudopotential file from pseudo\_dir. Rarely needed.& N\\
\hline
copy\_execs      & Bool & Whether to copy exectuables to the directory tree or use symbolic links instead. Default is True& N  \\
\hline
copy\_pseudos    & Bool & Whether to copy exectuables to the directory tree or use symbolic links instead. Default is True & N\\
\hline


\end{tabular}
%\end{tiny}
%% daemon & Defunct & \\
%% copyback\_every\_step & Defunct & \\
%% local\_scratch\_dir & Defunct & \\
%% local\_scratch & Defunct & \\

%    \hline
%\end{tabular}
\end{center}



\end{document}
